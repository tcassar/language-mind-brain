\documentclass[fontsize=12]{article}

\usepackage{spotted}

\title{House M.D. and Centres of the Brain}
\author{Week 9, Spotted in the Wild}
\date{Tom Cassar}

\begin{document}

\maketitle

\textit{This assignment \textbf{contains spoilers} for Season 2 Episodes 20 and 21 of 
House MD}. \linebreak


On April 1st, House MD was added to Netflix in the UK. As soon as I saw
this, I immediately started to binge watch. During Season 2
Episodes 20 and 21, I was reminded of this unit's content: specifically,
the areas of the brain responsible for language. \newline

\begin{figure}[H]
    \centering
    \includegraphics[width=0.8 \linewidth]{./img/house-foreman.png} 
    \caption{An image showing Dr. House and his colleague, Dr. Foreman. Taken
    from House MD.}
\end{figure}

House MD is an American medical sitcom starring Hugh Laurie who plays Dr
Gregg House, the eponymous hero. The sitcom follows the misanthropic doctor as
he goes about his day job, which is to diagnose and treat patients whose
symptoms baffle other doctors. \newline

Season 2 Episode 20 follows a cop who becomes infected with a mystery disease. The symptoms the cop experiences happen in distinct stages starting with euphoria, reduced motor function, blindness, and finally, hyperalgesia. \newline

The illness was caused by a parasite working its way through the
cop's brain. As it moved from physical brain centre to centre, the cop
experienced different symptoms. When the parasite was in the pain centre, the
cop developed hyperalgesia and did not respond to pain-relieving medication.

The cop never experienced any symptoms surrounding language. Watching the
episode made me wonder if the patient's ability to communicate would have
been affected by the parasite crossing Broca's area? Of course, House is 
not a medical resource, but I found it interesting nonetheless as it reminded 
me of \textit{Language, Mind and Brain}.

\end{document}


