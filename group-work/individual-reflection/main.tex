\documentclass[paper=a4,fontsize=16pt]{article}

\title{Individual Reflection}
\author{\textbf{Student ID} -- 11010030}
\date{\textbf{Group Number} -- 3}

\usepackage{mathptmx}
\usepackage[parfill]{parskip}
\usepackage{hologo}
\usepackage{siunitx}

\addtolength{\oddsidemargin}{-0.4in}
\addtolength{\evensidemargin}{-0.4in}
\addtolength{\textwidth}{0.8in}
\addtolength{\topmargin}{-1in}
\addtolength{\textheight}{1.0in}

\newcommand{\getwordcount}{\input{|"texcount -sum=1,0,0,0,0 -1 -q main.tex"}}

\begin{document}

\maketitle

Ammara, Hanifah, Tom, and I worked to create a
``YouTube-style'' video about \emph{modularity of mind} this semester. Our video
introduces the debate and presents arguments both for and against a
\textit{modular} versus an \textit{integrated} cognitive architecture.

\section{Work Distribution Across the Group}

Ammara scheduled a video call to discuss how we should split the work before
the first ``Group Check-In''. In the meeting, we decided to make four points
in the video --- two primary arguments, and two rebuttals. We each took
responsibility for \textit{one point}.

This allowed for an even split of work and made everyone accountable for an
even amount of academic research. Since our work was mostly disjoint, we
could each work at our own pace. In
turn, this allowed us to be more productive and made the experience more
pleasant. Despite our disjoint tasks, our WhatsApp group chat was active, with
questions receiving answers quickly. This continuous communication was important.

Our work was therefore split evenly across the group with everyone being
accountable for a given section. This clear accountability helped to keep
everyone on track  and avoided pressure around the deadline.

% Did your group encounter any challenges in the process? If so, how did
%your group deal with them?
\section{Group Challenges and Responses}

Our largest challenge as a group was finding time to meet/work which fit
everyone's schedule. In the end, meetings were often conducted without everyone
present. While this was not ideal, it did not negatively
affect the project. Missing group members always caught up on meeting using the
minutes and by asking on our WhatsApp group what work they must do.

Another challenge arose surrounding deadlines for other modules.  While most
of the impact was on me and my timetable, the group's response was important.
A friendly message on the group chat reminded me of editing footage that
needed to be done. This shows that we were able to adapt to things going
wrong. 

This happened around the time Tom and I had scheduled to produce the
conclusion. Tom, who saw that my workload was high, offered to take care of
the conclusion by himself. Tom's actions helped avoid me coming under
significant stress. After my deadline had passed, I was able to take some of
his work to ensure the overall workload remained balanced.

% Did you encounter any individual challenges in the process? If so, how did
% you deal with them?
\section{Individual Challenges and Responses}

As mentioned, I had my dissertation due on the Monday before submission. The pressure
around the deadline means that I had not edited footage by the time I  
agreed to have it done. However, after the friendly reminder from the group,
I edited the footage that evening.

After my deadline, I was able to pick up more work. As part of the editing
process, I cloned and compiled some open source code to allow us to pull the
``sine-wave sound effects'' from an existing YouTube video\footnote{Licenced
under the Standard YouTube Licence. I included 5 seconds out of a 35 second
video and is permissible under Fair Use}, leading to a higher quality video
than recording sounds from a browser. I organised our video shoot location
and edited Tom's footage to cover for him helping with the conclusion, and
made sure all referencing links were available to Ammara.

% Reflect on the collaborative process. What did you do well? What could
% you have done better? What do you think you can learn from this process?
\section{Reflection on the Collaborative Process}

Good group communication was key to producing a polished and well-researched video. 
The way the group handled issues also highlighted its strengths.

This group work experience taught me the importance of a trusting and coherent
\emph{group culture}. In past work, work has not been split evenly. People
have not always been active on our group chat, and criticism of work has
often been interpreted as personal attack. This friction was completely
absent from our group, with criticism always taken on board.

Personally, I could have been more proactive when problems presented themselves.
Collaboratively, I could have done a better job to keep up to date with work
that had already been completed. I often found papers relevant to my section,
but had been mentioned in our shared document. This wasted time which could have
otherwise been spent on other key areas (e.g. filming, scripting,
editing\dots).

Overall, I learned the value of a communicative and professional group. I also
learned a lot about the state of the literature covering modularity of mind.

\vspace{4em}
\begin{flushright}
    \rule{0.3\linewidth}{1pt}\par
    \vspace{0.5em}
    Word Count: \textbf{\getwordcount}
\end{flushright}

\end{document}

